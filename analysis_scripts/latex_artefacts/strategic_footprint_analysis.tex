Strategic footprint analysis reveals distinct behavioral patterns across LLM families, with model architecture proving far more influential than temperature settings in determining cooperative strategies. The analysis examines conditional cooperation probabilities based on previous round outcomes, comparing Anonymous Memory and Tracking Memory modes across four strategic scenarios.

\textbf{Model-Specific Cooperation Patterns:} Mistral-Medium exhibits the most consistently cooperative behavior across all scenarios, maintaining high cooperation rates even after being exploited (P(C|CD) ≈ 0.92-0.95) and in mutual defection situations (P(C|DD) ≈ 0.59-0.74). This suggests a robust "generous tit-for-tat" strategy that prioritizes relationship repair over immediate retaliation. In contrast, Gemini-2.0-Flash demonstrates the most punitive approach, with very low cooperation after mutual defection (P(C|DD) ≈ 0.05-0.19) and moderate forgiveness after exploitation (P(C|CD) ≈ 0.48-0.54), indicating a "grudging" strategic profile.

\textbf{Temperature Effects Are Negligible:} Across all models, temperature variations (T=0.2, 0.5, 0.7, 0.8, 1.2) produce minimal changes in cooperation probabilities, typically within 0.05-0.10 range. For example, Mistral-Medium's cooperation after mutual cooperation (P(C|CC)) varies only from 0.943 to 0.961 across temperature settings, while cooperation after exploitation (P(C|CD)) ranges from 0.916 to 0.950. This stability suggests that strategic reasoning patterns are deeply embedded in model architectures rather than being artifacts of sampling randomness.

\textbf{Memory Mode Impact:} Tracking Memory mode generally increases cooperative behavior compared to Anonymous Memory, particularly evident in Claude4-Sonnet's response to exploitation: P(C|CD) increases from 0.35-0.43 (Anonymous) to 0.67-0.78 (Tracking). This suggests that opponent identification enables more sophisticated reciprocal strategies. However, Mistral-Medium shows minimal memory mode sensitivity, maintaining consistently high cooperation regardless of tracking capability, reinforcing its inherently generous strategic disposition.

\textbf{Strategic Scenario Analysis:} All models maintain near-perfect cooperation after mutual cooperation (P(C|CC) > 0.95), confirming successful reciprocal relationship maintenance. The critical differentiation occurs in adversarial scenarios: Mistral-Medium's exceptional forgiveness (P(C|CD) > 0.90) contrasts sharply with Gemini's conditional cooperation (P(C|CD) ≈ 0.50) and Claude's moderate response (P(C|CD) ≈ 0.35-0.78). OpenAI models demonstrate intermediate patterns, with GPT5nano showing higher forgiveness than GPT4.1mini and GPT5mini. These distinct profiles suggest fundamental differences in how models balance exploitation vulnerability against relationship preservation.

\textbf{Exploitation Vulnerability vs. Strategic Sophistication:} The data reveals a tension between cooperation and strategic prudence. Mistral's high cooperation rates across all scenarios indicate either sophisticated long-term thinking or potential exploitation vulnerability. Gemini's selective cooperation suggests more calculated risk assessment, while Claude's memory-dependent behavior indicates adaptive strategic reasoning. These patterns have important implications for multi-agent LLM interactions and the design of AI systems requiring strategic decision-making.

\textbf{Extended Strategic Footprints Reveal Long-Term Reasoning Discrepancies:} The extended strategic footprints, which condition cooperation on the dominant outcome pattern across entire match histories rather than just the previous round, unveil significant behavioral inconsistencies that highlight the complexity of LLM strategic reasoning. Most notably, Claude4-Sonnet exhibits dramatic differences between immediate and historical context responses: while regular footprints show moderate forgiveness after exploitation (P(C|CD) ≈ 0.35-0.78), extended footprints reveal severely reduced cooperation in CD-dominated matches (P(C|CD=dom) ≈ 0.40-0.66), suggesting that repeated exploitation fundamentally alters Claude's strategic calculus. Even more striking, Claude's cooperation with successful exploiters (P(C|DC=dom)) drops to near zero in anonymous mode (0.01-0.11) but recovers substantially in tracking mode (0.0-0.50), indicating sophisticated opponent-specific grudge mechanisms. Conversely, Mistral-Medium shows remarkable consistency between immediate and extended contexts, maintaining high cooperation across both footprint types (P(C|CD) ≈ 0.92 vs P(C|CD=dom) ≈ 0.90), confirming its robustly generous strategic profile. Gemini-2.0-Flash demonstrates the most complex pattern: while showing moderate immediate forgiveness (P(C|CD) ≈ 0.50), it becomes significantly more punitive in exploitation-dominated matches (P(C|CD=dom) ≈ 0.24-0.48), suggesting that sustained patterns of exploitation trigger increasingly defensive responses. These discrepancies reveal that LLMs employ distinct temporal reasoning mechanisms, with some models (Claude) exhibiting sophisticated historical pattern recognition while others (Mistral) maintain consistent policies regardless of match context.